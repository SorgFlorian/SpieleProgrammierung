\section{Ergebnisse}
\label{sec:results}
%
In diesem Kapitel werden die Ergebnisse der Projektarbeit untersucht:
\begin{itemize}
	\item
		Stand des Prototyps (\fullref{sec:results/prototype})
	\item
		gewonnene Erfahrungen (\fullref{sec:results/xp})
	\item
		Ausblick und weitere M"oglichkeiten (\fullref{sec:results/future})
\end{itemize}
%
\subsection{Prototyp}
\label{sec:results/prototype}
%
Der Prototyp ist prinzipiell spielbar und zeigt das Prinzip der
Spielidee auf. Es sind mehrere Levels enthalten und das Starten
der Anwendung stellt den Spieler einem Hauptmen"u gegen"uber, in
welchem er das Spiel starten oder beenden kann. W"ahrend des Spiels
kann dieses pausiert werden.

Es sind im Laufe der Entwicklung viele Dateien hinzugekommen, die
nach jetzigem Stand "uberhaupt nicht verwendet werden; diese k"onnen
prinzipiell entfernt werden. Unity bietet hierbei jedoch scheinbar
keine Unterst"utzung; es ist nicht direkt ersichtlich, welche
Dateien gebraucht werden und welche nicht.

Zum Zeitpunkt der Abgabe besteht der Prototyp aus folgenden Komponenten:
\begin{itemize}
	\item Hauptmen"u
	\item Levelauswahlmen"u
	\item Pause Men"u
	\item F"unf Levels
	\begin{itemize}
		\item Level 1 : Sehr einfach; dazu gedacht, den Spieler in das Spielprinzip einzuf"uhren
		\item Level 2 : Schwerer als Level 1, allerdings noch recht kurz
		\item Level 3 : Dieses Level bietet erstmals die M"oglichkeit in einer Sackgasse zu landen.
			Befindet sich der Spieler in einer Sackgasse, hilft nur noch ein Neustart
		\item Level 4 : Dieses Level hat keine Sackgassen, ist allerdings l"anger und f"uhrt
			R"atselelemente ohne Schwerkraft"anderungen ein
		\item Level 5 : In diesem Level muss sich der Spieler zwischen verschiedenen Routen
			entscheiden (wobei nur eine zum Ziel f"uhrt) und Hindernissen in Form von
			Schwerkraftpfeilen ausweichen
	\end{itemize}
\end{itemize}

Neben den diesen nichtfunktionalen Komponenten besteht der Prototyp noch aus mehreren technischen
Aspekten, durch welche das Spielen erst m"oglich wird:
\begin{itemize}
	\item Animationen: Die verwendeten Animationen sind von Unity bereitgestellte MoCap Aufnahmen.
		Diese Aufnahmen mussten entsprechend auf den Avatar angepasst werden und mittels Blend Trees zum
		richtigen Zeitpunkt abgespielt werden.
	\item Input Handling: Das Input Handling (die Steuerung) wird gr"o"stenteils von Unity "ubernommen,
		mit ein paar Anpassungen bez"uglich der Schwerkraft"anderung
	\item Message Handling: Das Message Handling erlaubt es uns, auf bestimmte Nachrichten zu reagieren
			(neues Level laden, Pause, etc...)
	\item "Anderung der Schwerkraft: Das zentrale Spielkonzept; die Richtung der Schwerkraft kann sich "andern
	\item Kamera: Die Kameraf"uhrung funktioniert zwar, hat allerdings noch Raum f"ur Verbesserungen
\end{itemize}
%
\subsection{Erfahrungen}
\label{sec:results/xp}
%
Der Entwicklungsprozess zeigte deutlich, wie problematisch es
ist, wenn externe Softwarekomponenten benutzt werden, deren
Entwickler diese nicht f"ur den gew"unschten Zweck konzipiert haben.
Es waren so einige Anpassungen notwendig, um diese
``fremdprogrammierten'' Teile zielf"uhrend einsetzen zu k"onnen.

Die teilweise eher mangelhafte API-Referenz von Unity 3D zeigte
au"serdem einmal wieder auf, wie wichtig die umfassende Dokumentation
von Softwarekomponenten ist, welche von anderen Entwicklern
verwendet werden sollen.

Die teilweise verteilte Entwicklung und die gelegentlichen R"uckschritte
betonten die Notwendigkeit f"ur ein (ebenfalls verteiltes!)
Versionskontrollsystem\footnote{Subversion est mort. Vive Git!}.
%
\subsection{Ausblick}
\label{sec:results/future}
%
Obwohl wir insgesamt zufrieden mit unserem Prototyp sind, sehen wir noch einige
Verbesserungsm"oglichkeiten und Erweiterungspotential:

\begin{itemize}
	\item
		mehr Feedback f"ur den Benutzer (``Game Over''-Bildschirm etc.)
	\item
		qualitativere Modelle
	\item
		``Umsehen''-Funktion, um einen "Uberblick "uber die Karte
		zu erhalten
	\item
		mehr Puzzlecharakter f"ur die Levels
	\item Bessere Animationen: Obwohl unsere momentane Animationen funktionieren, gibt es auch hier viel
		Raum f"ur Verbesserungen, zum einen durch bessere Aufnahmen und zum anderen durch bessere Anpassung
		der Animationen an unseren Avatar und aneinander (zum Beispiel bessere "Uberg"ange zwischen
		Animationszust"anden).
	\item Bessere Kamerasteuerung: Im Moment ist einer der Hauptschwierigkeitsfaktoren unseres Spiels die
		Tatsache, dass die Kamerasteuerung sehr ungelenk ist. Zu w"unschen w"are hier eine freie Kamerasteuerung,
		damit sich der Spieler auf das Level und den Weg zum Ziel konzentrieren kann.
\end{itemize}
