\section{Ergebnisse}
\label{sec:results}
%
In diesem Kapitel werden die Ergebnisse der Projektarbeit untersucht:
\begin{itemize}
	\item
		Stand des Prototyps (\fullref{sec:results/prototype})
	\item
		gewonnene Erfahrungen (\fullref{sec:results/xp})
	\item
		Ausblick und weitere M"oglichkeiten (\fullref{sec:results/future})
\end{itemize}
%
\subsection{Prototyp}
\label{sec:results/prototype}
%
Der Prototyp ist prinzipiell spielbar und zeigt das Prinzip der
Spielidee auf. Es sind mehrere Levels enthalten und das Starten
der Anwendung stellt den Spieler einem Hauptmen"u gegen"uber, in
welchem er das Spiel starten oder beenden kann. W"ahrend des Spiels
kann dieses pausiert werden.

Es sind im Laufe der Entwicklung viele Dateien hinzugekommen, die
nach jetzigem Stand "uberhaupt nicht verwendet werden; diese k"onnen
prinzipiell entfernt werden. Unity bietet hierbei jedoch scheinbar
keine Unterst"utzung; es ist nicht direkt ersichtlich, welche
Dateien gebraucht werden und welche nicht.
%
\subsection{Erfahrungen}
\label{sec:results/xp}
%
Der Entwicklungsprozess zeigte deutlich, wie problematisch es
ist, wenn externe Softwarekomponenten benutzt werden, deren
Entwickler diese nicht f"ur den gew"unschten Zweck konzipiert haben.
Es waren so einige Anpassungen notwendig, um diese
``fremdprogrammierten'' Teile zielf"uhren einsetzen zu k"onnen.

Die teilweise eher mangelhafte API-Referenz von Unity 3D zeigte
au"serdem einmal wieder auf, wie wichtig die umfassende Dokumentation
von Softwarekomponenten ist, welche von anderen Entwicklern
verwendet werden sollen.

Die teilweise verteilte Entwicklung und die gelegendlichen R"uckschritte
betonte die Notwendigkeit f"ur ein (ebenfalls verteiltes!)
Versionskontrollsystem\footnote{Subversion est mort. Vive Git!}.
%
\subsection{Ausblick}
\label{sec:results/future}
%
Sollte das Konzept weitergef"uhrt werden, so w"ahren noch einige
Verbesserungen einzubauen:

\begin{itemize}
	\item
		mehr Feedback f"ur den Benutzer (``Game Over''-Bildschirm etc.)
	\item
		qualitativere Modelle
	\item
		Es sollte wahrscheinlich noch eine bessere Semantik
		der Kameraf"uhrung gefunden werden.
	\item
		``Umsehen''-Funktion, um einen "Uberblick "uber die Karte
		zu erhalten
	\item
		mehr Puzzlecharakter f"ur die Levels
\end{itemize}
