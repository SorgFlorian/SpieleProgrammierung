\section{Umsetzung}
\label{sec:implementation}
%
Dieses Kapitel beschreibt die Umsetzung der einzelnen Aspekte des
entwickelten Computerspiels.
%
\subsection{Kartengestaltung}
\label{sec:implementation/maps}
%
Der Einfachheit halber sind im Prototyp die Karten aus Quadern aufgebaut.
Es sollte keine gro"se Schwierigkeit darstellen, ``bessere'' Modelle
zu verwenden, sofern solche vorhanden sind. Durch die "ubliche Kollisionsdetektion
kann der Spieler ohne weiteres Zutun auf den Oberfl"achen der Quader laufen.
Das Ziel (oder die Ziele; also jedlicher Ort, dessen Erreichen einen Level als
``gewonnen'' beendet) ist ebenfalls ein solcher ``normaler'' Bestandteil der Karte
und unterscheidet sich lediglich durch den Game Object Tag \texttt{targetZone}
von ``Nicht-Zielen''. Durch die Pr"asenz dieses Tags ended das Spiel bei
ber"uhren des entsprechenden Objekts (im Prototyp ebenfalls Quader).

Falls der Spieler von dem gegebenen Parcour herunterf"allt, soll er nicht
endlos fallen, soldern das Spiel ebenfalls (als ``verloren'') beendet
werden. Hierzu wird jeder Level im l"uckenlose W"ande eingeschlossen,
deren ber"uhren den Spieler ``t"otet''. Dies passert, weil diese W"ande
den Game Object Tag \texttt{killZone} besitzen.

Die Effekte dieser beiden Tags sind in \fullref{sec:implementation/avatar}
n"aher beschrieben.
%
\subsection{Schwerkraft"anderung}
\label{sec:implementation/gravity}
%
Die aufsammelbaren Schwerkraftpfeile (siehe \fullref{sec:implementation/arrows})
zeichnen sich durch den Game Object Tag \texttt{gravArrow} aus. Beim Ber"uhren
eines Objekts mit diesem Tag wird die Schwerkraft gem"a"s den Informationen in
der Komponente \texttt{GravitationOrientation} des Objekts (welche es auch
besitzen muss) ver"andert. Das Objekt wird hierbei deaktiviert, kann also danach
nicht noch einmal aufgesamment werden. Die Klasse \texttt{PlayerCtlBehavior}
erledigt dies in ihrer Methode \texttt{OnTriggerEnter}, welche von Unity
aufgerufen wird, wenn der Besitzer der betreffenden Instanz der Klasse (die
Spielfigur) ein anderes Objekt ber"uhrt. Listing
\ref{lst:implementation/gravity/OnTriggerEnter} zeigt desen Vorgang.

\begin{lstlisting}[caption={Kollision mit Schwerkraftpfeil},label=lst:implementation/gravity/OnTriggerEnter]
if (other.gameObject.tag == "gravArrow") {
    other.gameObject.SetActive (false);
    GravitationOrientation orientation = other.gameObject
            .GetComponent<GravitationOrientation> ();
    targetGrav = new Vector3 (
        orientation.x_Grav_Factor,
        orientation.y_Grav_Factor,
        orientation.z_Grav_Factor
    );
    boost = orientation.boostNeeded;
}
\end{lstlisting}

Die zentralen Eingenschaften von \texttt{GravitationOrientation} sind
die Attribute \texttt{?\_Grav\_Factor}, welche die Komponenten des Vektor darstellen,
der die neue Schwerkraft angibt.

Wenn sich die Richtung der Schwerkraft "andert, so wird die Spielfigur
gedreht, so dass die gem"a"s der neuen Richtung ``aufrecht'' steht
(d.h. mit den F"u"sen in der Richtung der neuen Schwerkraft). Diese
Drehung kann Probleme verursachen, da Teilen der Spielfigur ein Teil
der Levelgeometrie ``im Weg'' sein kann. Um dies zu verhindern, besitzt
\texttt{GravitationOrientation} zus"atzlich das Flag \texttt{boostNeeded}.
Falls dieses \texttt{true} ist, so wird die Spielfigur nach "Anderung
der Schwerkraft und Drehen der Figur noch nach oben geschleudert, um
sich von der Oberfl"ache zu entfernen. Listing
\ref{lst:implementation/gravity/gravityUpdate} zeigt diesen Vorgang
in der Klasse \texttt{PlayerCtlBehavior}.

\begin{lstlisting}[caption={Anpassung der Schwerkraft und Spielfigur},label=lst:implementation/gravity/gravityUpdate]
// Schwerkraftanpassung
bool gravityUpdate() {
    if (Physics.gravity.normalized != targetGrav) {
        Physics.gravity = targetGrav * 9.81f;
        return true;
    }
    return false;
}

// Physik-Update
void FixedUpdate() {
    if (gravityUpdate ()) {
        transform.up = -Physics.gravity; // Figur drehen
        // Figur anhalten
        GetComponent<Rigidbody>().velocity
                = Vector3.zero;
        // Figur von Quader wegbewegen
        if (boost)
            transform.position += Physics.gravity
                    .normalized * boostStrength;
    }
    // ...
}
\end{lstlisting}
%
\subsection{Spielfigur}
\label{sec:implementation/avatar}
Das Modell der Spielfigur ist dem Asset Store entnommen.
Mit den in Unity enthaltenen Klassen \texttt{ThirdPersonCharacter}
und \texttt{ThirdPersonUserControl} kann der Spieler diese
bereits steuern (siehe \fullref{sec:implementation/controls}).

Als zus"atzliche Komponente f"ur die Spielfigur existiert
die Komponentenklasse \texttt{PlayerCtlBehavior}, welche die Effekte
des Ber"uhrens von speziellen Objekten definiert. Die
Interaktion mit den Schwerkraftpfeilen ist in
\fullref{sec:implementation/gravity} beschrieben.

Falls der Spieler zu Tode st"urzt, sprich ein Objekt mit dem
Tag \texttt{killZone} ber"uhrt, so wird der Level neu geladen,
wodurch der Spieler wieder an die Ausgangsposition zur"uckversetzt
wird. Listing \ref{lst:implementation/avatar/killZone}
zeigt diesen Vorgang.

\begin{lstlisting}[caption={Sterben},label=lst:implementation/avatar/killZone]
void OnTriggerEnter(Collider other) {
    // ...
    if (other.gameObject.tag == "killZone") {
        killed = true;
    }
    // ...
}

void FixedUpdate() {
    // ...
    if(killed) {
        killed = false;
        Application.LoadLevel(Application.loadedLevel);
    }
}
\end{lstlisting}

Falls der Spieler das Ziel erreicht, sprich ein Objekt mit dem
Tag \texttt{targetZone} ber"uhrt, so wird der n"achste Level geladen
(oder das Men"u, falls kein n"achster Level existiert). Listing
\ref{lst:implementation/avatar/targetZone} zeigt diesen
Vorgang.

\begin{lstlisting}[caption={Gewinnen},label=lst:implementation/avatar/targetZone]
void OnTriggerEnter(Collider other) {
    // ...
    if (other.gameObject.tag == "targetZone") {
        int nextLevel = Application.loadedLevel + 1;
        if (nextLevel < Application.levelCount)
            Application.LoadLevel(nextLevel);
        else
            Application.LoadLevel(0);
    }
}
\end{lstlisting}
%
\subsection{Steuerung}
\label{sec:implementation/controls}
%
Die Reaktionen der Spielfigur auf Steuerungstasten werden
von der in Unity enthaltenen Klasse \texttt{ThirdPersonCharacter}
definiert. Problematisch hierbei ist, dass die Autoren dieser
Klasse nicht damit gerechnet haben, dass an W"anden oder der
Decke entlang gelaufen wird. Es sind entsprechend so einige
Stellen in dem Standard-Code enthalten, welche die positive
Y-Achse als ``oben'' betrachten. Da sich durch die
Schwerkraft"anderungen die Bedeutung von ``oben'' unvermittelt
"andern kann, funktioneren diese Definitionen nicht direkt
in der ``Standardausf"uhrung''.

Es mussten dementsprechend alle Bez"uge auf hartcodierte
Richtungen durch Ausdr"ucke ersetzt werden, welche die
gew"unschte Richtung anhand der derzeitigen Schwerkraft
berechnen. Listing
\ref{lst:implementation/controls/hardcodedY} zeigt solch
einen Fall.

\begin{lstlisting}[caption={Korrektur hartcodierter Richtungen},label=lst:implementation/controls/hardcodedY]
// Falsch: Unity nimmt +Y als oben an
m_Animator.SetFloat("Jump", m_Rigidbody.velocity.y);
// Richtig: Die Richtung der Schwerkraft ist entscheidend
m_Animator.SetFloat("Jump", (m_Rigidbody.velocity
    - Vector3.ProjectOnPlane(m_Rigidbody.velocity,
    -Physics.gravity)).magnitude);
\end{lstlisting}
%
\subsection{Kamera}
\label{sec:implementation/camera}
%
Anf"anglich wurde die Kamera auf klassische Art hinter
der Spielfigur plaziert. Hierdurch verdeckt die Figur
allerdings den vor ihr liegenden Teil der Szene, was
oft nicht sinnvoll ist. Daher wird stattdessen der
Unity-eigene Mechanismus der ``Multipurpose Camera Rig''
verwendet, welcher die Kamera der Spielfigur so folgen
l"asst, dass sich die Kamera hierzu m"oglichst wenig
bewegen muss. Der Effekt hiervon entspricht in etwa
einem Kameramann, der die Szene filmt, aber immer
die Spielfigur im Bild beh"alt.
%
\subsection{Schwerkraftpfeile}
\label{sec:implementation/arrows}
%
F"ur die Schwerkraftpfeile wurde in Blender eigens ein
Modell eines dreidimensionalen Pfeils erstellt. Die
Pfeile zeichnen sich durch den Game Object Tag
\texttt{gravArrow} aus (siehe \fullref{sec:implementation/maps}).
Die Schwerkraft"anderung selbst wird von der Spielfigur
durchgef"uhrt, nicht von den einzelnen Pfeilen
(siehe \fullref{sec:implementation/gravity}).
%
\subsection{Game Over}
\label{sec:implementation/gameover}
Sowohl das Gewinnen als auch das Verlieren den Spiels
entspricht dem Ber"uhren spezieller Objekte (Ziel
bzw. Kill Zone). Beide F"alle sind bis jetzt lediglich
durch das Laden von Levels implementiert (siehe
\fullref{sec:implementation/avatar}). In einem
``echten'' Produkt sollte vermutlich Feedback in Form
einer Nachricht gegeben werden.
%
\subsection{Men"us und Levelfolge}
\label{sec:implementation/menus}
%
Die Men"us wurden mit den ``2D-Elemente im 3D-Raum''-F"ahigkeiten
von Unity erstellt. Der Einfachheit halber bestehen sie schlicht
aus einfachen Buttons. Das Klicken der Buttons wird von der
Klasse \texttt{HandleMessages} mit passenden Reaktionen verbunden.
Im Prototyp m"ussen zu Testzwecken die Levels nicht ``freigespielt''
werden, sondern k"onnen auch im Menu direkt angew"ahlt werden.
