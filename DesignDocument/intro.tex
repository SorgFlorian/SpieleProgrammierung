\section{Einf"uhrung}
\label{sec:intro}
%
Dieses Kapitel befasst sich mit den organisatorischen Aspekten des Projekts:
\begin{itemize}
	\item
		Worum geht es "uberhaupt? (\fullref{sec:intro/requirements})
	\item
		Was wurde erstellt? (\fullref{sec:intro/idea})
	\item
		Womit wurde es erstellt? (\fullref{sec:intro/technologies})
\end{itemize}
%
\subsection{Aufgabenstellung}
\label{sec:intro/requirements}
%
Gem"a"s des Manuskripts des Kurses war als Projektarbeit die ``Konzeption und Entwicklung
eines Computerspiels'' vorgesehen, mit einem spielbaren ``Prototypen'' als Ergebnis. Die
verwendeten Technologien sowie die Natur des Spiels selbst waren offen gelassen. Es galt
also zun"achst eine Idee zu finden und diese dann umzusetzen. Im Zuge dieser Umsetzung
galt es, sich f"ur passende Technologien zu entscheiden, um die Implementation des
Prototyps zu erm"oglichen.
%
\subsection{Idee}
\label{sec:intro/idea}
Der Ansatz zur Ideenfindung war, einem der ``Klassiker'' der Videospielwelt neues
Leben einzuhauchen, indem man einen ``Twist'' (also eine unerwartete ``Wende'';
in diesem Fall eine Eigenheit des Spielmechanik) hinzuf"ugt. Die Wahl fiel auf
ein klassisches Jump and Run -- der naheliegende Twist ist daher, die Grundlage eben
dieses Laufens und Springens zu beeinflussen. Der wohl grundlegendste Aspekt hiervon
ist die Schwerkraft.

Aus dieser Logik ergab sich der Twist, die Richtung der Schwerkraft w"ahrend des
Spielablaufes zu ver"andern. Da eine solche "Anderung wahrscheinlich eher frustrierend
ist, falls sie nicht der Kontrolle des Spielers unterliegt, sollte diesem lediglich
die \textit{M"oglichkeit} gegeben werden, diese Kontrolle auszu"uben. Die Herausforderung
verschiebt sich hierdurch: Anstatt gegen die Schwerkraft"anderung zu ``k"ampfen'',
muss der Spieler diese geschickt einsetzten, um seine Ziele zu erreichen.

Um die Kontrolle "uber die Schwerkraft dem Spieler zu "uberlassen, aber diese
Kontrolle dennoch einzuschr"anken, wurde folgendes Konzept erdacht: In jedem
Level sind im Voraus Pfeile plaziert, welche als aufsammelbare ``Items''
fungieren. Wenn der Spieler einen solchen Pfeil aufsammelt, so "andert sich
die Schwerkraft in die Richtung, in welche der Pfeil zeigt. Beim Kartendesign
k"onnen somit Routen geschaffen werden, welche das Aufsammeln gewisser Pfeile
beinhalten.

Auf der Basis dieses Konzept w"urde dann das naheliegende Ziel des Spiels das
Erreichen eines gewissen Ortes sein; der Spieler muss dann einen Weg zu diesem Ort
finden. Durch die Pr"asenz der Pfeile kann dieser Weg das Entlanglaufen an
W"anden oder der Decke beinhalten.
%
\subsection{Verwendete Technologien}
\label{sec:intro/technologies}
Als bereits existierende Game Engine wurde Unity 3D gew"ahlt. Dies scheint auf
den ersten Blick insbesondere durch den eingebauten ``Asset Store'' attraktiv,
da dort Modelle, Animationen, Texturen und andere Artefakte bereits vorgefertigt
(und teilweise auch umsonst) zur Verf"ugung stehen. Die Spielelogik wurde
dementsprechend in der Programmiersprache C\# geschrieben.

Soll der Prototyp weiterentwickelt oder eingesehen werden, wird daher Unity 3D
ben"otigt. Durch die eingebaute Distribution des Mono-Projekts sind sowohl
Entwicklungswerkzeuge (Toolchain; insbesondere Compiler und Linker) f"ur C\#
dort bereits enthalten. Mit MonoDevelop enth"alt Unity 3D auch bereits eine
Entwicklungsumgebung f"ur Microsoft .NET (und damit auch C\#).
