\documentclass[landscape,compress,table]{beamer}
\usepackage[utf8]{inputenc}
\usepackage[ngerman]{babel}
\usepackage{xcolor}
\usetheme{Warsaw}
\usefonttheme{professionalfonts}
\useoutertheme[footline=authortitle]{miniframes}
\setbeamertemplate{navigation symbols}{}%remove navigation symbols
\setbeamertemplate{blocks}[rounded][shadow=true]
\title{Spieleprogrammierung -- Endpräsentation}
\author{Simon Bausch, Florian Sorg}
\institute{Hochschule Aalen}
\newcommand*\oldmacro{}%
\let\oldmacro\insertshorttitle%
\renewcommand*\insertshorttitle{%
	\oldmacro\hfill%
	\insertframenumber\,/\,\inserttotalframenumber}
\begin{document}

\begin{frame}
	\titlepage
\end{frame}

\begin{frame}
	\tableofcontents
\end{frame}

\section{Idee}
\begin{frame}
	\begin{itemize}
		\item
			3D Jump \& Run
		\item
			Twist: Aufsammelbare Objekte "andern die
			Richtung der Schwerkraft
		\item
			Engine: Unity\\
			Spielelogik in C\#
	\end{itemize}
\end{frame}

\section{Herausforderungen}
\begin{frame}
	\begin{itemize}
		\item
			Unity scheint nicht auf jedem Computer
			zu laufen
		\item
			Mehrere Entwickler\\
			$\mapsto$ problematisch, da viele Dateien
			Bin"arformate aufweisen und teilweise
			Versionsnummern in Dateien gespeichert
			werden (Unity-Projekt...)
		\item
			Dokumentation der API teilweise ungenau
			(vom API-Design selbst gar nicht zu reden;
			Euler-Winkel? Oh, boy...)
		\item
			Schwerkraft"anderung h"ort sich einfach
			an, \textbf{aber}...
	\end{itemize}
\end{frame}

\begin{frame}
	\begin{itemize}
		\item
			Bewegung der Spielfigur sowie Kameraf"uhrung
			ist im 3D-Raum nicht gerade trivial.
			Vorgefertigte ``Assets'' versprechen Abhilfe,
			\textbf{aber}...
		\item
			In den Programmschnipseln der Assets ist
			oft +Y als ``oben'' hartkodiert.\\
			$\mapsto$ "Anderungen der Standard-Assets
			erforderlich
		\item
			Bewegung ist aus Kamerasicht (analog zu
			einem gewissen italienischen Klempner);
			das ``folge dem Spieler'' Ka\-me\-ra\-mo\-dul
			ist aber ganz offensichtlich nicht
			daf"ur gedacht, dass der Spieler an der
			Wand entlang l"auft...
	\end{itemize}
\end{frame}

\section{Demo}
\begin{frame}
	\begin{center}
		\Huge
		Demo
	\end{center}
\end{frame}

\begin{frame}
	\begin{center}
		\Large
		Vielen Dank f"ur Ihre Aufmerksamkeit!
	\end{center}
\end{frame}

\end{document}
