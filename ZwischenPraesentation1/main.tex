\documentclass[landscape,compress,table]{beamer}
\usepackage[utf8]{inputenc}
\usepackage[ngerman]{babel}
\usepackage{xcolor}
\usetheme{Warsaw}
\usefonttheme{professionalfonts}
\useoutertheme[footline=authortitle]{miniframes}
\setbeamertemplate{navigation symbols}{}%remove navigation symbols
\setbeamertemplate{blocks}[rounded][shadow=true]
\title{Spieleprogrammierung -- Zwischenpräsentation 1}
\author{Simon Bausch, Florian Sorg}
\institute{Hochschule Aalen}
\newcommand*\oldmacro{}%
\let\oldmacro\insertshorttitle%
\renewcommand*\insertshorttitle{%
	\oldmacro\hfill%
	\insertframenumber\,/\,\inserttotalframenumber}
\begin{document}

\begin{frame}
	\titlepage
\end{frame}

\begin{frame}
	\tableofcontents
\end{frame}

\section{Spielidee}
\begin{frame}
	\begin{itemize}
		\item
			Genre: 3D Jump \& Run
		\item
			Twist: Einsammeln von Items "andert die
			Richtung der Schwerkraft
		\item
			R"atsel/Geschicklichkeit -- Nicht herunterfallen, etc...
		\item
			Ziel: Finde den Weg zu einem bestimmten Punkt.
		\item
			eventuell mehrere L"osungswege
		\item
			Die Story muss noch ausgearbeitet/ausgedacht werden.
	\end{itemize}
\end{frame}

\section{Eingesetzte Software}
\begin{frame}
	\begin{itemize}
		\item Engine: Unity 3D
		\item Vorteile:
			\begin{itemize}
				\item
					Einige Models, Animationen, ...
					sind vordefiniert im ``Asset Store''
					verf"ugbar.
				\item ``kostenlos''
				\item platformunabh"angig
				\item gut dokumentiert (Google)
			\end{itemize}
		\item Nachteile:
			\begin{itemize}
				\item ``Dont call us, we'll call you'' -- wenig Kontrolle
				\item gro"ser Overhead
				\item Performance k"onnte besser sein
			\end{itemize}
	\end{itemize}
\end{frame}

\section{Zeitplan}
\begin{frame}
	\begin{tabular}{|l|l|}
		\hline
		20.03. - 24.04. & Ideenfindung und Konzepterarbeitung\\
		\hline
		24.04. & erste Zwischenpr"asentation\\
		\hline
		24.04. - 05.06. & feinere Ausarbeitung, erste Prototypen\\
		\hline
		04.06. & \textbf{Alpha}\\
		\hline
		05.06. & zweite Zwischenpr"asentation\\
		\hline
		05.06. - 03.07. & Implementierung\\
		\hline
		02.07. & \textbf{Beta}\\
		\hline
		03.07. & Abschlusspr"asentation\\
		\hline
		16.07. & Abgabe\\
		\hline
	\end{tabular}
\end{frame}

\section{Demo}
\begin{frame}
	\begin{center}
		Demo
	\end{center}
\end{frame}

\begin{frame}
	\begin{center}
		Vielen Dank f"ur Ihre Aufmerksamkeit!
	\end{center}
\end{frame}

\end{document}
