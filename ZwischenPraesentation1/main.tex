\documentclass[landscape,compress,table]{beamer}
\usepackage[utf8]{inputenc}
\usepackage[ngerman]{babel}
\usepackage{xcolor}
\usetheme{Warsaw}
\usefonttheme{professionalfonts}
\useoutertheme[footline=authortitle]{miniframes}
\setbeamertemplate{navigation symbols}{}%remove navigation symbols
\setbeamertemplate{blocks}[rounded][shadow=true]
\title{Projektarbeit f\"ur Vorlesung Spieleprogrammierung}
\author{Simon Bausch}
\institute{Hochschule Aalen}
\newcommand*\oldmacro{}%
\let\oldmacro\insertshorttitle%
\renewcommand*\insertshorttitle{%
	\oldmacro\hfill%
	\insertframenumber\,/\,\inserttotalframenumber}
\begin{document}

\begin{frame}
	\titlepage
\end{frame}

\begin{frame}
	\tableofcontents
\end{frame}

\section{Gruppenmitglieder}
\begin{frame}
	TODO
\end{frame}

\section{Idee}
\begin{frame}
	\begin{itemize}
		\item
			XXX -- with a twist!
		\item
			Genre: Jump \& Run
		\item
			Twist: Einsammeln von Items "andert die
			Richtung der Schwerkraft
	\end{itemize}
\end{frame}

\section{Eingesetzte Software}
\begin{frame}
	\begin{itemize}
		\item
			Engine: Unity 3D
		\item
			So einige Models, Animationen, ...
			sind vordefiniert im ``Asset Store''
			verf"ugbar.
	\end{itemize}
\end{frame}

\section{Zeitplan}
\begin{frame}
	TODO
\end{frame}

\end{document}
